\documentclass[a4paper,10pt]{article}
\pagestyle{empty}
\usepackage{hyperref}
\usepackage{palatino}
\usepackage[american,USenglish]{babel}
\usepackage{graphicx}
\usepackage[leqno]{amsmath}
\usepackage[margin=1.5cm,vmargin={0pt,1cm},includefoot]{geometry}
\title{}
\author{Mark Johnson \\
Loyola University Chicago}
\date {mjohnson4@luc.edu}
% End of preamble,
% the body now follows
\begin{document}
\maketitle

\section{Problem 1B}
\emph{Consider a pool of six I/O buffers. Assume that any buffer is just as likely to be available (or occupied) as 
any other. Compute the probability associated with the following event: \newline
B = "At least three but not more than five buffers occupied."} [Trivendi, 1982]

\subsection{Step 1: (Identify the sample space S)}
The points in the chosen sample space must be mutually exclusive and collectively exhaustive. The former means that 
no two points in the sample space can occur simultaneously. The latter means that the outcome from any trial must be 
exactly one point from the sample space. We are counting the number of ways that we can select k objects from among n 
objects where order is important and the same object can be repeated any number of times. Because the sample space 
can be defined as a problem of ordered samples of size k with replacements (permutations with replacements), and 
because each sample point represents one of these permutations, we assert that the sample space has $ n^{k} $ or $ 2 
^{6} $ sample points. We assert that the sample space defined below is mutually exclusive and collectively 
exhaustive. \newline
$ S = \{s_{0}, s_{1}, s_{2},...,ss_{63}\} $ where \newline
$ s_{0} = \{0,0,0,0,0,0\}, s_{1} = \{0,0,0,0,0,1\}, s_{2} = \{0,0,0,0,1,0\}, s_{3} = \{0,0,0,0,1,1\},...,s_{63} = 
\{1,1,1,1,1,1\} $

\subsection{Step 2: (Assign probabilities)}
Because the problem statement specifies that each event is equiprobable, each sample point has probability of $ 1/S $ 
or 1/64.

\subsection{Step 3: (Identify the events of interest)}
B = "At least three but no more than five buffers are occupied."

\subsection{Step 4: (Compute desired probabilities)}
\end{document} X
