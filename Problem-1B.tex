\documentclass[a4paper,10pt]{article}
\pagestyle{empty}
\usepackage{hyperref}
\usepackage{palatino}
\usepackage[american,USenglish]{babel}
\usepackage{graphicx}
\usepackage[leqno]{amsmath}
\usepackage[margin=1.5cm,vmargin={0pt,1cm},includefoot]{geometry}
\title{}
\author{Mark Johnson \\
Loyola University Chicago}
\date {mjohnson4@luc.edu}
% End of preamble,
% the body now follows
\begin{document}
\maketitle

\section{Problem 1B}
\emph{Consider a pool of six I/O buffers. Assume that any buffer is just as likely to be available (or occupied) as 
any other. Compute the probabilities associated with the following events: \newline
B = "At least three but not more than five buffers occupied."} [Trivendi, 1982]

\subsection{Step 1: (Identify the sample space S)}
The points in the chosen sample space must be mutually exclusive and collectively exhaustive. The former means that 
no two points in the sample space can occur simultaneously. The latter means that the outcome from any trial must be 
exactly one point from the sample space. We are counting the number of ways that we can select k objects from among n 
objects where order is important and the same object can be repeated any number of times. Because the sample space 
can be defined as a problem of ordered samples of size k with replacements (permutations with replacements), and 
because each sample point represents one of these permutations, we assert that the sample space has $ n^{k} $ or $ 2 
^{6} $ sample points. We assert that the sample space defined below is mutually exclusive and collectively 
exhaustive. \newline
$ S = \{s_{0}, s_{1}, s_{2},...,ss_{63}\} $ where \newline
$ s_{0} = \{0,0,0,0,0,0\}, s_{1} = \{0,0,0,0,0,1\}, s_{2} = \{0,0,0,0,1,0\}, s_{3} = \{0,0,0,0,1,1\},...,s_{63} = 
\{1,1,1,1,1,1\} $

\subsection{Step 2: (Assign probabilities)}
Because the problem statement specifies that each event is equiprobable, each sample point has probability of $ 1/S $ 
or 1/64.

\subsection{Step 3: (Identify the events of interest)}
B = "At least three but no more than five buffers are occupied."

\subsection{Step 4: (Compute desired probabilities)}
$ B_{1} = $ "Exactly three buffers are occupied." \newline
The cardinality of $ B{1} $ will be equal to $ n!/(k!(n-k)!) $, because this combinatorial function computes the 
number 
of unordered samples of size k without replacements. This will compute the number of ways that you can choose 3 
unordered points from 6 points, without replacements. I am saying that there are 6 buffers and we can only 
choose three without replacement. Therefore: \newline
The cardinality of $ B_{1} = 6!/(3!*(6-3)!) $ \newline
The cardinality of $ B_{1} = 720/(6*6) $ \newline
The cardinality of $ B_{1} = 720/36 $ \newline
The cardinality of $ B_{1} = 20 $ \newline
Therefore, the cardinality of $ B_{1} $ is 20. \newline
The cardinality of $ B_{2} = $ "Exactly four buffers are occupied." \newline
The cardinality of $ B_{2} = 6!/(4!*(6-4)!) $ \newline
The cardinality of $ B_{2} = 720/(24*2) $ \newline
The cardinality of $ B_{2} = 720/48 $ \newline
The cardinality of $ B_{2} = 15 $ \newline
Therefore, the cardinality of $ B_{2} $ is 15. \newline
$ B_{3} = $ "Exactly five buffers are occupied." \newline
The cardinality of $ B_{3} = 6!/(5!*(6-5)!) $ \newline
The cardinality of $ B_{3} = 720/(120*1) $
The cardinality of $ B_{3} = 6 $ \newline
Therefore, the cardinality of $ B_{3} $ is 6. \newline
$ B = {B_{1}, B_{2}, B_{3}} $ \newline
$ B = B_{1} \cup B_{2} \cup B_{3} $ \newline
$ P(B) = $ The cardinality of B divided by the cardinality of S. \newline
$ P(B) = (20+15+6)/64 $ \newline
$ P(B) = 41 /64 $ \newline
Therefore, the probability that three but no more than five buffers are occupied equals 41/64.
\end{document} X
