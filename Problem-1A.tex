\documentclass[a4paper,10pt]{article}
\pagestyle{empty}
\usepackage{hyperref}
\usepackage{palatino}
\usepackage[american,USenglish]{babel}
\usepackage{graphicx}
\usepackage[leqno]{amsmath}
\usepackage[margin=1.5cm,vmargin={0pt,1cm},includefoot]{geometry}
\title{}
\author{Mark Johnson \\
Loyola University Chicago}
\date {mjohnson4@luc.edu}
% End of preamble,
% the body now follows
\begin{document}
\maketitle
\section{Example 1.1 Problem}
\emph{Consider a pool of siz I/O buffers. Assume that any buffer is just as likely to be available (or occupied) as 
any other. Compute the probabilities associated with the following event: \newline
A = "At least two but no more than five buffers occupied."} [Trivendi, 1982]
\subsection{Step 1: (Identify the sample space S)}
The points in the chosen sample space must be mutually exclusive and collectively exhaustive.The former means that no 
two points in the sample can occur simultaneously. The latter means that the outcome of any trial must be exactly one 
point from the sample space. We are counting the number of ways that we can select k objects from among n objects 
where order is important and the same object can be repeated any number of times. Because the sample space can be 
defined as a problem of ordered samples of size k with replacements (permutations with replacements), and because 
each sample point represents one of these permutations, we assert that the sample space has $ n^{k} $ or $ 2^{6} $ 
sample points. We assert that the sample space defined below is mutually exclusive and collectively exhaustive. 
\newline
$ S = \{s_{0}, s_{1}, s_{2},..., s_{63}\} $ where \newline
$ s_{0} = \{0,0,0,0,0,0\}, s_{1} = \{0,0,0,0,0,1\}, s_{2} = \{0,0,0,0,1,0\}, s_{3} = \{0,0,0,0,1,1\},..., s_{63} = 
\{1,1,1,1,1,1\} $
\subsection{Step 2: (Assign probabilities)}
Because the problem statement specifies that each event is equiprobable, each sample point has probability of $ 1/S $ 
or 1/64.
\subsection{Step 3: (Identify the events of interest)}
A = "At least two but no more than five buffers are occupied."
A = "Two, three, four, or five buffers are occupied."
By the complementation laws: \newline
The complement of A, $ \bar{A} $ = "Zero, one, or six buffers are 
occupied"
$ \bar{A} = \{(0,0,0,0,0,0), (0,0,0,0,0,1), (0,0,0,0,1,0), \newline
 (0,0,0,1,0,0), (0,0,1,0,0,0), (0,1,0,0,0,0), (1,0,0,0,0,0), (1,1,1,1,1,1)\} $ \newline
$ \bar{A} = \{s_{0}, s_{2}, s_{4}, s_{8}, s_{16}, s_{32}, s_{63}\} $
\subsection{Step 4: (Compute desired probabilities)}
$ \bar{A} = s_{0} \cup s_{1} \cup s_{2} \cup s_{4} \cup s_{8} \cup s_{16} \cup s_{32} \cup s_{63} $ \newline
By axiom(A3'), \newline
$ P(\bar{A}) = P(s_{0}) + P(s_{1}) + P(s_{2}) + P(s_{4}) + P(s_{8}) + P(s_{16}) + P(s_{32}) + P(s_{63}) $ \newline
Because each point in the sample space is equiprobable and because there are 64 points in the sample space, \newline
$ P(\bar{A}) = 1/64 + 1/64 + 1/64 + 1/64 + 1/64 + 1/64 + 1/64 + 1/64 $ \newline
$ P(\bar{A}) = 8/64 $ \newline
$ P(\bar{A}) = 1/8 $ \newline
By the complementation law of events, $ A \cup \bar{A} = S. $ \newline
By axiom(A3'), $ P(A \cup \bar{A}) = P(A) + P(\bar{A}). $ \newline
By axiom(A2), $ P(S) = 1. $ \newline
Therefore, $ P(A) + P(\bar{A}) = 1. $ \newline
And, $ P(A) = 1 - P(\bar{A}). $ \newline
Substituting, $ P(A) = 1 - 1/8. $ \newline
Finally, $ P(A) = 7/8. $ \newline 
\end{document} X
