\documentclass[a4paper,10pt]{article}
\pagestyle{empty}
\usepackage{hyperref}
\usepackage{palatino}
\usepackage[american,USenglish]{babel}
\usepackage{graphicx}
\usepackage[leqno]{amsmath}
\usepackage[margin=1.5cm,vmargin={0pt,1cm},includefoot]{geometry}
\title{}
\author{Mark Johnons \\
luther07@gmail.com}
\date {(312)479-2894}
% End of preamble,
% the body now follows
\begin{document}
\maketitle
\section{Example 1.1}
\subsection{Step 1: (Identify the sample space S)}
$ S = \{s_{0}, s_{1}, s_{2},..., s_{31}\} $, with $ s_{0} = \{0,0,0,0,0\}, s_{1} = \{0,0,0,0,1\}, s_{2} = \{0,0,0,1,0\}, s_{3} = \{0,0,0,1,1\},..., s_{31} = 
\{1,1,1,1,1,\} $, where 0 means busy and 1 means available.
\subsection{Step 2: (Assign probabilities)}
Because the problem statement doesn't specify a probability distribution, we assume that each $ s_{i} $ is 
equiproable with probability of 1/32.
\subsection{Step 3: (Identify the events of interest)}
E = "the probability that a process is scheduled for immediate execution, given that the process needs at least 
three tape drives for its execution." \newline
$ E = \{(0,0,1,1,1), (0,1,0,1,1), (0,1,1,0,1), (0,1,1,1,0), (0,1,1,1,1), (1,0,0,1,1), (1,0,1,0,1), (1,0,1,1,0), 
(1,0,1,1,1), (1,1,0,0,1), (1,1,0,1,0), (1,1,0,1,1), (1,1,1,0,0), (1,1,1,0,1), (1,1,1,1,0), (1,1,1,1,1)\} $ \newline
$ E = \{s_{4}, s_{11}, s_{13}, s_{14}, s_{15}, s_{19}, s_{21}, s_{22}, s_{23}, s_{25}, s_{26}, s_{27}, s_{28}, 
s_{29}, s_{30}, s_{31}\} $
\subsection{Step 4: (Compute desired probabilities)}
\end{document} X
