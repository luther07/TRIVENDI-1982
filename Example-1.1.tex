\documentclass[a4paper,10pt]{article}
\pagestyle{empty}
\usepackage{hyperref}
\usepackage{palatino}
\usepackage[american,USenglish]{babel}
\usepackage{graphicx}
\usepackage[leqno]{amsmath}
\usepackage[margin=1.5cm,vmargin={0pt,1cm},includefoot]{geometry}
\title{}
\author{Mark Johnson \\
Loyola University Chicago}
\date {mjohnson4@luc.edu}
% End of preamble,
% the body now follows
\begin{document}
\maketitle
\section{Example 1.1 Problem}
\emph{Consider an example of a computer system with five identical tape drives. One possible random experiment 
consists of checking the system to see how many tape drives are currently available. Each tape drive is in one of 
two states: busy (labeled 0) and available (labeled 1). An outcome of the experiment (a point in the sample space) 
can be denoted by a 5-tuple of 0's and 1's. A 0 in position i of the 5-tuple indicates that tape drive i is busy and 
a 1 indicates that it is available. The sample space has $ 2^{5}=32 $ sample points, as shown in the table on page 8. 
Assume that we are required to determine the probability that a process is scheduled for immediate execution, given 
that the process needs at least three tape drives for its execution.} [Trivendi, 1982]
\subsection{Step 1: (Identify the sample space S)}
The points in the chosen sample space must be mutually exclusive and collectively exhaustive. The former means that 
no two points in the sample space can occur simultaneously. The latter means that the outcome of any trial must be 
exactly one point from the sample space. We are counting the number of ways that we can select k objects from among n 
objects where order is important and the same object can be repeated any number of times. Because the sample space 
can be defined as a problem of ordered samples of size k with replacements (permutations with replacements), and 
because each sample point represents one of these permutations, we assert that the sample space has $ n^{k} $ or $ 2^{5} $ sample points. We assert that the sample 
space defined below is mutually exclusive and collectively exhaustive. \newline
$ S = \{s_{0}, s_{1}, s_{2},..., s_{31}\} $ where \newline
$ s_{0} = \{0,0,0,0,0\}, s_{1} = \{0,0,0,0,1\}, s_{2} = \{0,0,0,1,0\}, s_{3} = \{0,0,0,1,1\},..., s_{31} = \{1,1,1,1,1,\} $
\subsection{Step 2: (Assign probabilities)}
Because the problem statement doesn't specify a probability distribution, we assume that each $ s_{i} $ is 
equiproable with probability of $ 1/S $ or 1/32.
\subsection{Step 3: (Identify the events of interest)}
E = "the probability that a process is scheduled for immediate execution, given that the process needs at least 
three tape drives for its execution." \newline
$ E = \{(0,0,1,1,1), (0,1,0,1,1), (0,1,1,0,1), (0,1,1,1,0), \newline
 (0,1,1,1,1), (1,0,0,1,1), (1,0,1,0,1), (1,0,1,1,0), \newline 
(1,0,1,1,1), (1,1,0,0,1), (1,1,0,1,0), (1,1,0,1,1), \newline
(1,1,1,0,0), (1,1,1,0,1), (1,1,1,1,0), (1,1,1,1,1)\} $ \newline
$ E = \{s_{7}, s_{11}, s_{13}, s_{14}, s_{15}, s_{19}, s_{21}, s_{22}, s_{23}, s_{25}, s_{26}, s_{27}, s_{28}, 
s_{29}, s_{30}, s_{31}\} $
\subsection{Step 4: Compute desired probabilities}
$ P(E) = P(s_{7} \cup s_{11} \cup s_{13} \cup s_{14} \cup s_{15} \cup s_{19} \cup s_{21} \cup s_{22} \cup s_{23} 
\cup s_{25} \cup s_{26} \cup s_{27} \cup s_{28} \cup s_{29} \cup s_{30} \cup s_{31}) $ \newline
By axiom(A3'), \newline
$ P(E) = P(s_{7}) + P(s_{11}) + P(s_{13}) + P(s_{14}) + P(s_{15}) + P(s_{19}) + P(s_{21}) +
P(s_{22}) + P(s_{23}) + P(s_{25}) + P(s_{26}) + P(s_{27}) + P(s_{28}) + P(s_{29}) + P(s_{30}) + P(s_{31}) $ \newline
Because each point in the sample space is equiprobable and because there are 32 points in the sample space, 
\newline
$ P(E) = 1/32 + 1/32 + 1/32 + 1/32 + 1/32 + 1/32 + 1/32 + 1/32 + 1/32 + 1/32 + 1/32 + 1/32 + 1/32 + 1/32 + 1/32 
+ 1/32 $ \newline
$ P(E) = 16/32 $ \newline
$ P(E) = 1/2 $ \newline
The probability of immediate execution in any trial is 0.50.
\end{document} X
